%math
\usepackage{a4wide}
\usepackage{float}
\usepackage{graphicx, latexsym, longtable}
\usepackage{epsfig, color, psfrag}
\usepackage{amsmath, amsfonts, amssymb}
\usepackage{multirow}
\usepackage{fancyvrb}
\usepackage{subfig}
\usepackage{xcolor}
\usepackage{CJKutf8}
\captionsetup[subfigure]{labelformat=empty}
\usepackage{indentfirst}
\usepackage{listings}
\usepackage{nccmath}

% Kopf- und Fußzeilen
\pagestyle{fancy}
\fancyhf{}
\fancyhead[EL,OR]{\sffamily\thepage}
\fancyhead[ER,OL]{\sffamily\leftmark}

\fancypagestyle{headings}{}

\fancypagestyle{plain}{}

\fancypagestyle{empty}{
  \fancyhf{}
  \renewcommand{\headrulewidth}{0pt}
}

% Kein "Kapitel # NAME" in der Kopfzeile
\renewcommand{\chaptermark}[1]{
	\markboth{#1}{}
   	\markboth{\thechapter.\ #1}{}
}

% Zeilenabstand: 1,5
\onehalfspacing 

% Akzentfarbe
\definecolor{akzent}{RGB}{133,60,46}

% Schriften
\newfontfamily\corpoSfamily{Helvetica}
\newfontfamily\corpoAfamily{Helvetica}
\setmainfont{Helvetica}

\addtokomafont{chapter}{\corpoAfamily\LARGE} 
\addtokomafont{section}{\corpoSfamily\Large\color{akzent}} 
\addtokomafont{subsection}{\corpoSfamily\large\mdseries} 
\addtokomafont{subsubsection}{\corpoSfamily\normalsize\mdseries}
\addtokomafont{caption}{\corpoSfamily\normalsize\mdseries} 

% Schriften im IHV
\renewcommand{\cftchapfont}{\sffamily\normalsize}
\renewcommand{\cftsecfont}{\sffamily\normalsize}
\renewcommand{\cftsubsecfont}{\sffamily\normalsize}
\renewcommand{\cftchappagefont}{\sffamily\normalsize}
\renewcommand{\cftsecpagefont}{\sffamily\normalsize}
\renewcommand{\cftsubsecpagefont}{\sffamily\normalsize}

% Zeilenabstand in den Verzeichnissen einstellen
\setlength{\cftparskip}{.5\baselineskip}
\setlength{\cftbeforechapskip}{.1\baselineskip}

% Einrücken von Absätzen deaktivieren
\setlength{\parindent}{0pt}

% Listings
\definecolor{codegreen}{rgb}{0,0.6,0}
\definecolor{codegray}{rgb}{0.5,0.5,0.5}
\definecolor{codepurple}{rgb}{0.58,0,0.82}
\definecolor{backcolour}{rgb}{0.95,0.95,0.92}

\definecolor{listinggray}{gray}{0.9}
\definecolor{lbcolor}{rgb}{0.9,0.9,0.9}
\lstset{
backgroundcolor=\color{lbcolor},
    tabsize=4,    
%   rulecolor=,
    language=[GNU]C++,
        basicstyle=\scriptsize,
        upquote=true,
        aboveskip={1.5\baselineskip},
        columns=fixed,
        showstringspaces=false,
        extendedchars=false,
        breaklines=true,
        prebreak = \raisebox{0ex}[0ex][0ex]{\ensuremath{\hookleftarrow}},
        frame=single,
        numbers=left,
        showtabs=false,
        showspaces=false,
        showstringspaces=false,
        identifierstyle=\ttfamily,
        keywordstyle=\color[rgb]{0,0,1},
        commentstyle=\color[rgb]{0.026,0.112,0.095},
        stringstyle=\color[rgb]{0.627,0.126,0.941},
        numberstyle=\color[rgb]{0.205, 0.142, 0.73},
%        \lstdefinestyle{C++}{language=C++,style=numbers}’.
}
\lstset{
    backgroundcolor=\color{lbcolor},
    tabsize=4,
  language=C++,
  captionpos=b,
  tabsize=3,
  frame=lines,
  numbers=left,
  numberstyle=\tiny,
  numbersep=5pt,
  breaklines=true,
  showstringspaces=false,
  basicstyle=\footnotesize,
%  identifierstyle=\color{magenta},
  keywordstyle=\color[rgb]{0,0,1},
  commentstyle=\color{codegreen},
  stringstyle=\color{red}
  } 

%\lstdefinestyle{codestyle}{
%	backgroundcolor=\color{backcolour},   
%    commentstyle=\color{codegreen},
%    keywordstyle=\color{magenta},
%    numberstyle=\tiny\color{codegray},
%    stringstyle=\color{codepurple},
%    basicstyle=\footnotesize,
%    breakatwhitespace=false,         
%    breaklines=true,                 
%    captionpos=b,                    
%    keepspaces=true,                 
%    numbers=left,                     
%    numbersep=5pt,                  
%    showspaces=false,                
%    showstringspaces=false,
%    showtabs=false,                  
%    tabsize=2
%}
%
%\lstset{style=codestyle}
%
%\lstdefinelanguage{JavaScript}{
%	keywords={break, case, catch, continue, debugger, default, delete, do, else,
%	finally, for, function, if, in, instanceof, new, return, switch, this, throw,
%	try, typeof, var, void, while, with}, morecomment=[l]{//},
%	morecomment=[s]{/*}{*/}, morestring=[b]',
%	morestring=[b]",
%	sensitive=true
%}

% include für Visio Dokumente. 1cm Rand wird weggeschniten
\newcommand{\includevisio}[2][]{\includegraphics[clip, trim=1cm 1cm 1cm 1cm, #1]{#2}} 
